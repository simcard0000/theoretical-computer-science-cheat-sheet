The Chinese remainder theorem:
There exists a number $C$ such that:
\vskip 0pt
$$
\eqalign{
C &\equiv r_{1} \bmod m_{1} \cr
&\vdots \quad \vdots \quad \vdots \cr
C &\equiv r_{n} \bmod m_{n} \cr
}
$$
if $m_{i}$ and $m_{j}$ are relatively prime for $i\neq j$.
\vskip 3pt
\par
Euler's function:
$\phi(x)$ is the number of positive integers less than $x$ relatively prime to $x$.
If $\prod_{i=1}^n p^{e_i}_i$ is the prime factorization of $x$ then
$$\phi(x) = \prod_{i=1}^n p^{e_i - 1}_i (p_i - 1).$$
Euler's theorem:
If $a$ and $b$ are relatively prime then
$$ 1 \equiv a^{\phi(b)} \bmod b. $$
Fermat's theorem:
$$ 1 \equiv a^{p-1} \bmod p. $$
The Euclidean algorithm:
if $a > b$ are integers then
$$\gcd(a, b) = \gcd(a \bmod b, b).$$
If $\prod_{i=1}^n p^{e_i}_i$ is the prime factorization of $x$ then
$$S(x) = \sum_{d\vert x} d = \prod_{i=1}^n {p^{e_i+1}_i - 1 \over p_i - 1}.$$
Perfect Numbers: $x$ is an even perfect number iff $x = 2^{n-1}(2^n - 1)$ and $2^n - 1$ is prime.
Wilson's theorem: $n$ is a prime iff
$$
(n-1)! \equiv -1 \bmod n.
$$
M\"obius inversion:
$$
\mu(i) = \cases{
1 &if $i = 1$. \cr
0 &if $i$ is not square-free. \cr
(-1)^r &if $i$ is the product of\cr
&$r$ distinct primes. \cr
}
$$
If
$$
G(a) = \sum_{d \vert a} F(d),
$$
then
$$
F(a) = \sum_{d \vert a} \mu(d) G\Big({a \over d}\Big).
$$
Prime numbers:
$$
\eqalign{
p_n &= n \ln n + n \ln \ln n - n + n {\ln \ln n \over \ln n}\cr
 &\qquad {} + O\bigg({n \over \ln n}\bigg), \cr
\pi(n) &= {n \over \ln n} + {n \over (\ln n)^2} + {2! n \over (\ln n)^3} \cr
 &\qquad {} + O\bigg({n \over (\ln n)^4}\bigg).\cr
}
$$
